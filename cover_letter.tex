\documentclass[12pt,english]{article}
\usepackage[T1]{fontenc}
\usepackage[latin9]{inputenc}
\usepackage{geometry}
\geometry{verbose,tmargin=3cm,bmargin=3.4cm,lmargin=3cm,rmargin=3.4cm}
\usepackage{setspace}
\onehalfspacing
\usepackage{babel}
\begin{document}
\begin{center}
\textbf{\LARGE{Response to the Reviews of MissingDataGUI}}
\bigskip{}
\par\end{center}{\LARGE \par}

The substantial changes were made on both the package and the manuscript.
The main changes of the package include:

\begin{itemize}
\item embedding two more packages (mice and mi) for multiple imputation;
\item adding a new imputation method based on the nearest neighbors;
\item enabling multiple tabs in the graphics widget, to compare different
imputation methods and multiple imputation chains;
\item adding a setting tab in the GUI to adjust the parameters and the imputation
methods of mice;
\item allowing the input variables to be the ordered factors.
\end{itemize}

The main changes in the article include:
\begin{itemize}
\item introduction and comparison among the current multiple imputation
packages in R;
\item example of how to check the missing assumptions;
\item more background of the plot types;
\item explicit explanation on the imputation methods of the nearest neighbors;
\item comparison among the multiple imputation chains.
\end{itemize}
\clearpage{}


\section*{Response to Review 1}
\begin{enumerate}

\item What action should the user take in response to the generated plots?
(Para. 1 \& 6)
\begin{itemize}
\item In our revision, the examples are explained more explicitly in details
with some conclusions and suggestion. Also, we introduced how to use
the GUI to check the missing pattern assumptions like MCAR.
\end{itemize}

\item There are problems in reproducing Figure 5. (Para. 2)
\begin{itemize}
\item We do not have any problem in reproducing Figure 5. MissingDataGUI
has released a new version, so hopefully the problem could be fixed.
\end{itemize}

\item Ordinal variables are omitted. (Para. 4)
\begin{itemize}
\item Ordinal variables are accepted now. Median could be obtained as the
center after sorted. Methods provided by the package mice can be used
for the ordinal variables too.
\end{itemize}

\item It is not clear what happens when the user conditions on a factor
that itself has missingness. (Para. 5)
\begin{itemize}
\item This question is answered in the revision. The package was improved
to separate the observations missing on the conditioning factor.
\end{itemize}

\item Missingness map: what should the user do with the positive or negative
association? (Para. 7)
\begin{itemize}
\item We use a new example to illustrate the missingness map, which shows
a positive association. The reason of the association in the example
is explained and the corresponding suggestion is provided.
\end{itemize}

\item In section 3.2, the authors choose an imputation algorithm to maximize
the similarity, however, it is neither necessary nor sufficient to
indicate whether the imputations are reasonable (Para. 8)
\begin{itemize}
\item Theoretically, with some assumptions on the data we may compare the
imputation methods and suggest a reasonable one. But for a real dataset
without any assumption and the true values for the missings, there
is no numerical or graphical criteria to indicate which method is
better. For section 3.2, ``maximize the similarity to indicate a
reasonable imputation'' is not the most accurate summary. We matched
the distribution of imputed values with the observations, to provide
more understanding of the methods, and the conditioning factor. From
the plots, it is reasonable to caution the conditional-median method,
compared to the multiple imputation result.
\end{itemize}

\item The package supports so many imputation algorithms that are known
to have poor theoretical properties. Why are the others supported,
which invariably means that some users will utilize them? (Para. 9)
\begin{itemize}
\item The simple imputation methods like overall mean and median, are practical
and useful under some situations, like the exploratory analysis, or
displaying the missings conditional on the known clusters.
\item A couple of imputation methods were added to the package, including
two multiple imputation packages: mice and mi.
\end{itemize}

\item The parallel coordinates plot could be used to visualize whether the
completed data set is consistent with the missing-at-random (MAR)
assumption. (Para. 10)
\begin{itemize}
\item An example was added to Section 3 on how to use the parallel coordinates
plot and other plots to check the missing assumptions. MCAR (missing
completely at random) can be checked, but MAR is not testable. However,
it is possible to suggest the dependent variables via the plots.
\end{itemize}

\end{enumerate}
\clearpage{}

\section*{Response to Review 2}
\begin{enumerate}

\item It does not even clearly warn against using univariate imputation
approaches for multivariate analysis purposes. (Para. 2)
\begin{itemize}
\item This issue cannot be answered easily by adding a warning under that
situation. This package is designed for the novice users to explore
and understand the missing structure by trying various imputation
methods. Since the users could load any types of data into the GUI,
we have no idea if the multivariate analysis is appropriate, or if
the variables are independent in a multivariate analysis. For any
real data, users usually start from an exploration, before knowing
which impuation method to use. The univariate methods are handy and
practical at the beginning of an analysis. A warning saying that the
univariate imputation is risky may not be correct all the time, and
we do not want to prevent the users from trying different methods
than what they ``should'' use.
\end{itemize}

\item How the variables used in regression imputation are chosen? (Para.
2)
\begin{itemize}
\item The variables can be selected in area 1 of Figure 1. All of the selected
variables are used during the imputing. The regression method is a
multiple imputation method provided by the package Hmisc, and renamed
to ``MI: areg'' in the new version.
\end{itemize}

\item How several imputations from multiple imputation are used in plotting?
(Para. 2)
\begin{itemize}
\item In the revision we allow the users to pick the number of imputation
sets in any multiple imputation. Different sets are plotted in different
tabs.
\end{itemize}

\item For the less well-known types of plot, e.g. the spine plot, a brief
description should be given. (Para. 2)
\begin{itemize}
\item Descriptions and references are given for most of the plot types in
the revision.
\end{itemize}

\item Some other related packages are completely ignored (Para. 3)
\begin{itemize}
\item Now they are briefly introduced in Section 1, and the comparison of
the multiple imputation packages is given in Section 2.3.
\end{itemize}

\item Installation issues (Para. 4, Bullet 1)
\begin{itemize}
\item The installation of Gtk+ on Windows is not fun since Windows users
may need to edit the environment variable PATH manually. Once you
get this step done, it should be easy to install RGtk2 and gWidgetsRGtk2
in R.
\end{itemize}

\item Pairwise plot does not work on Windows R 3.0.1. (Para. 4, Bullet 2)
\begin{itemize}
\item MissingDataGUI has released a new version 0.2-0, the problem should
be fixed. In a recent test we didn't have any problem for the pairwise plot.
\end{itemize}

\item Warning message ``ggpcp is deprecated''. (Para. 4, Bullet 3)
\begin{itemize}
\item Fixed in version 0.2-0 of the package.
\end{itemize}

\item When creating a numerical summary of missingness for the complete
dataset, why do I have to select all variables? (Para. 5, Bullet 1)
\begin{itemize}
\item Now if no variables are selected, then the numerical summary will
include all the variables. Shift + left-click will make it easy to
select all.
\end{itemize}

\item Why doesn\textquoteright{}t conditioning affect the numerical summary?
(Para. 5, Bullet 2)
\begin{itemize}
\item Fixed in version 0.2-0.
\end{itemize}

\item In the parallel coordinate plot, switch plotting order so that the
missings are displayed on top of the non-missings. (Para. 5, Bullet
3)
\begin{itemize}
\item Fixed in version 0.2-0.
\end{itemize}

\item As many people are red-green blind, choosable colors or at least a
better choice of predefined colours would be helpful. (Para. 5, Bullet
4)
\begin{itemize}
\item We changed the color scheme to make it colorblind friendly.
\end{itemize}

\end{enumerate}

\end{document}
