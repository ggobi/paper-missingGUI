%% LyX 2.0.6 created this file.  For more info, see http://www.lyx.org/.
%% Do not edit unless you really know what you are doing.
\documentclass[english]{article}
\usepackage[T1]{fontenc}
\usepackage[latin9]{inputenc}
\usepackage{geometry}
\geometry{verbose,lmargin=3.4cm,rmargin=3.6cm}
\usepackage{babel}
\begin{document}

\title{Response to the Reviews of MissingDataGUI}

\maketitle
Minor revisions were made to both the manuscript and package.
\begin{itemize}
\item Changes in the manuscript include:

\begin{itemize}
\item statement on the limitation of using univariate/single imputation
methods
\item description of the median and mean imputation
\item comparison between multiple imputation methods (especially for aregImpute
in Hmisc)
\item argument on the data generation process in accessing MCAR or MAR
\item improving the language
\end{itemize}
\item Changes in the package are:

\begin{itemize}
\item renaming the ``Mean/Mode'' method to ``Mean''
\item fixing a few bugs
\end{itemize}
\end{itemize}

\subsection*{Response to the AE report}

A main issue from both reviewers and the associate editor was with the simple
imputation methods. We agree that single imputation ``understates
the uncertainty in the imputation'' (Reviewer \#1), and if the variance
and covariance estimation is crucial to the analysis then one should
not use single imputation or univariate imputation. To inform readers,
in the paper we removed the sentence in page 7 about the simple imputation method, and added a paragraph that gives more details on the top of page 8. We stated
that the univariate imputation methods ignore the dependencies between
variables and produce biased variance and covariance estimation.
And we used Figure 4 as an example to show the inadequacy. However,
we do not agree that the methods are always wrong under any assumption
or any circumstance. In a few applications that we have seen, when results are not impacted by the simple imputation, it is ok to use them. The paper now puts more emphasis on better imputation methods.

The second major issue was that the description of median imputation
was confusing.  We reworded the description, and made a small change
in the software.  The mode was used when the type of variable
prevented the calculation of the mean and median, and allowed for
seamless operations for the user, in order to continue exploring the
missing patterns.  The names of simple imputation methods are now
``median'', ``mean'', and ``random value''. We explained in page 7
that the software makes automatic choices and when the mode is returned.

We updated the comparison between multiple imputation methods, and
emphasized that it is important to think about the data generation
process when checking MCAR and MAR, as required by Reviewer
\#2. (Details below in the response to Review 2.)

We also followed the AE's suggestion to make some small changes:
\begin{itemize}
\item New heading for the last section. Instead of ``Discussion'', we
use ``Summary''.
\item Correct citation to XGobi, GGobi, and MANET are made. We checked
  with the authors to these software to make sure the references are
  correct.
\item Rewording paragraphs and sentences throughout the paper to improve the language.
\end{itemize}

\subsection*{Response to Review 1}

Reviewer \#1 questioned the median method and single imputation. We
have answered them in the response to the AE. One other thing that
reviewer \#1 mentioned is that the package froze RStudio. We cannot
reproduce the same problem -- it may not come from this package. To
diagnose will need more tests (and user reports) among various
versions of software and platform. We will keep an eye on it and
update the package as necessary, and in response to user feedback.


\subsection*{Response to Review 2}

Reviewer \#2 addressed a couple of questions. The serious one is still
about univariate/single imputation, and we answered it in the response
to the AE. Remaining questions are answered as follows.
\begin{itemize}
\item The multiple imputation methods are default to three imputed data
sets, and the reviewer suggests to default to one only.

\begin{itemize}
\item We set the default number of sets to be 3, because we want to emphasize
that multiple imputation methods are supposed to give multiple sets
of result, and the difference between the sets reveals the variation.
Taking the number back to 1 will make the user ignore the uncertainty.
So we did not make any change here.
\end{itemize}
\item The description of the multiple imputation packages is sloppy and
misleading -- ``sounds as though the Hmisc package uses a very inflexible
selection of models''.

\begin{itemize}
\item The paragraph is revised to make the statement clearer. However,
  Reviewer \#2 misunderstood what ``flexible'' means. Hmisc provides
  three models to predict the missings, while mice and mi provide
  nearly 10 models. Moreover, they allow for the user to write
  their own model, as we mentioned in Table 2. In our opinion, the
  user-specific models make the package mice and mi more
  ``flexible''. But Reviewer \#2 is correct that Hmisc automatically
  transforms the quantitative variables with a user-specific
  weight, to get the imputed value from the predicted value. We
  mentioned this in the revision.
\end{itemize}
\item Plots in Figure 8 are too small, and need better comments.

\begin{itemize}
\item We enlarged the size of the plots.
\end{itemize}
\item Should mention how multiply imputed data are exported.

\begin{itemize}
\item The export of multiple imputation results is explained in the first
paragraph of page 15.
\end{itemize}
\item Checking of assumptions should mention thinking about the data generation
process.

\begin{itemize}
\item The data generation process is mentioned now in the first paragraph
of page 19.\end{itemize}
\end{itemize}

\end{document}
