%% LyX 2.0.6 created this file.  For more info, see http://www.lyx.org/.
%% Do not edit unless you really know what you are doing.
\documentclass[english]{article}
\usepackage[T1]{fontenc}
\usepackage[latin9]{inputenc}
\usepackage{geometry}
\geometry{verbose,lmargin=3.4cm,rmargin=3.6cm}
\usepackage{babel}
\begin{document}

\title{Response to the Reviews of MissingDataGUI}

\maketitle
A minor revision was made to both the manuscript and package.
\begin{itemize}
\item Changes in the manuscript include:

\begin{itemize}
\item statement on the limitation of using univariate/single imputation
methods
\item description of the median and mean imputation
\item comparison between multiple imputation methods (especially for aregImpute
in Hmics)
\item argument on the data generation process in accessing MCAR or MAR
\item tweaking the sentences and words.
\end{itemize}
\item Changes in the package are:

\begin{itemize}
\item renaming the ``Mean/Mode'' method to ``Mean''
\item fixing a few bugs.
\end{itemize}
\end{itemize}

\subsection*{Response to the AE report}

A main issue from both reviewers and the editor is with the simple
imputation methods. We agree that the single imputation ``understate
the uncertainty in the imputation'' (Reviewer \#1), and if the variance
and covariance estimation is crucial to the analysis then one should
not use single imputation or univariate imputation. To give a warning
in the paper, we dropped the argument in page 7 about the median or
mean method, and then added a paragraph on the top of page 8. We stated
that the univariate imputation methods ignore the dependencies between
variables and produce bias in the variance and covariance estimation.
And we used Figure 4 as an example to show the inadequacy. However,
we do not agree that the methods are always wrong under any assumption
or any circumstance. In some extreme situation, or when the variance/covariance
estimation is not cared much, we would like to give the user some
freedom to explore the data with simple methods.

The next issue is about the confusing description of median imputation.
We reworded the description, and made a small change in the software.
Actually we use the mode in the initial design because the mean and
median methods do not work for some types of variables, and we do
not want that interrupts the easiest exploration. After the change,
the names of simple imputation methods are ``median'', ``mean'',
and ``random value''. We explained in page 7 that the software makes
automatic choices and when the mode is returned.

We updated the comparison between multiple imputation methods, and
mentioned to think about the data generation process when checking
MCAR and MAR, as required by Reviewer \#2. (Details below in the response
to Review 2.)

We also followed the editor's suggestion to make some little changes:
\begin{itemize}
\item New heading for the last section. Instead of ``Discussion'', we
use ``Summary''.
\item Correct citation to XGobi, GGobi, and MANET. We checked with the authors
to these software to make sure the references are correct.
\item Rewording some paragraphs and sentences.
\end{itemize}

\subsection*{Response to Review 1}

Reviewer \#1 questioned the median method and single imputation. We
have answered them in the response to the AE. One other thing that
reviewer \#1 mentioned is that the package froze RStudio. We cannot
reproduce the same problem -- it may not come from the package. To
find out the real problem will need more tests (and user reports)
among various versions of software and platform. We will keep an eye
on it and updating the package if necessary.


\subsection*{Response to Review 2}

Reviewer \#2 addressed a couple of questions. The serious one is still
about univariate/single imputation, and we answered it in the response
to the AE. The rest part of the questions are answered as follows.
\begin{itemize}
\item The multiple imputation methods are default to three imputed data
sets, and the reviewer suggests to default to one only.

\begin{itemize}
\item We set the default number of sets to be 3, because we want to emphasize
that multiple imputation methods are supposed to give multiple sets
of result, and the difference between the sets reveals the variation.
Taking the number back to 1 will make the user ignore the uncertainty.
So we did not make any change here.
\end{itemize}
\item The description of the multiple imputation packages is sloppy and
misleading -- ``sounds as though the Hmisc package uses a very inflexible
selection of models''.

\begin{itemize}
\item The paragraph is revised to make the statement clear. However, Reviewer
\#2 misunderstood what ``flexible'' means. Hmisc provides three
models to predict the missings, while mice and mi provide nearly 10
models, moreover, they keep a port for the user to write their own
model, as we mentioned in Table 2. In our opinion, the user-specific
models make the package mice and mi more ``flexible''. But Reviewer
\#2 is correct that Hmisc uses automatic transformation of the quantitative
variables with a user-specific weight, to get the imputed value from
the predicted value. We mentioned this in the revision.
\end{itemize}
\item Plots in Figure 8 are too small, and need better comments.

\begin{itemize}
\item We enlarged the size of the Figure.
\end{itemize}
\item Should mention how multiply imputed data are exported.

\begin{itemize}
\item The export of multiple imputation methods is explained in the first
paragraph of page 15.
\end{itemize}
\item Checking of assumptions should mention thinking about the data generation
process.

\begin{itemize}
\item The data generation process is mentioned now in the first paragraph
of page 19.\end{itemize}
\end{itemize}

\end{document}
